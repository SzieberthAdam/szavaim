\startenvironment e_szotaram

% https://tex.stackexchange.com/a/43411/50554
% https://www.gammon.com.au/scripts/doc.php?lua=table.insert
\startluacode

  userdata = userdata or {}
  userdata.bibrefgyujto_lista = {}
  userdata.hivatkozasgyujto_lista = {}

  function userdata.hivatkozasgyujto_torol()
    userdata.bibrefgyujto_lista = {}
    userdata.hivatkozasgyujto_lista = {}
  end

  function userdata.bibrefgyujto_ad(szoveg)
    userdata.bibrefgyujto_lista [#userdata.bibrefgyujto_lista + 1] = szoveg
  end

  function userdata.hivatkozasgyujto_ad(szoveg)
    userdata.hivatkozasgyujto_lista [#userdata.hivatkozasgyujto_lista + 1] = szoveg
  end

  function userdata.hivatkozasgyujto_ki()
    if (next(userdata.hivatkozasgyujto_lista) ~= nil or next(userdata.bibrefgyujto_lista) ~= nil) then
      for key, value in ipairs(userdata.hivatkozasgyujto_lista) do
        userdata.hivatkozasgyujto_lista[key] = "\\hbox{" .. value .. "}"
      end
      for key, value in ipairs(userdata.bibrefgyujto_lista) do
        userdata.bibrefgyujto_lista[key] = "\\hbox{" .. value .. "}"
        print(userdata.bibrefgyujto_lista[key])
      end
      context("\\quad\\hfill")
      if next(userdata.hivatkozasgyujto_lista) ~= nil then
        context("$\\parallel$~{\\tfx\\em " .. table.concat(userdata.  hivatkozasgyujto_lista,"; ") .. "}")
      end
      if (next(userdata.hivatkozasgyujto_lista) ~= nil and next(userdata.bibrefgyujto_lista) ~= nil) then
        context("\\quad")
      end
      if next(userdata.bibrefgyujto_lista) ~= nil then
        context("$\\tfx\\rightarrow$~{\\tfx\\em " .. table.concat(userdata.  bibrefgyujto_lista,"; ") .. "}")
      end
      userdata.hivatkozasgyujto_torol()
    end

  end

\stopluacode




\definecounter[SzoBekSzam][way=bytext,prefix=no]
\definecounter[AxiomaSzam][way=bytext,prefix=no]

\def\Szotextcommand#1{\hbox{#1\enspace\bullet\enspace}}

\definehead[Szo][section]
% https://tex.stackexchange.com/a/375930/50554
%\setuphead [Szo][%S
%  number=no,%
%  %textdistance=0pt,%
%  %commandbefore={\resetcounter[ParagraphNumber]\null},%
%  %textcommand=\Szotextcommand,%
%  commandafter={\enspace\bullet\enspace},%
%  %commandafter={\hskip 0.5em plus 0.25em minus 0.125em\relax\bullet\hskip 0.25em plus 0.25em minus 0.125em\relax},
%  %commandafter={\hskip 0.5em plus 0.25em minus 0.125em\relax\bullet\kern 0.25em\relax},
%  alternative=text,%
%  insidesection={\resetcounter[SzoBekSzam]},%
%  style=\bf,%
%  textdistance=-\currentspaceskip%
%]

\definefloat[szo][szavak]
\setupfloats[szo][sidealign=line]

% \define[2]\MyChapter
%   {\framed[frame=on,location={inner,none,low},height=2\baselineskip,align=flushleft]{#2}}

\setuphead[Szo][%S
  number=no,%
  %before=\relax,% ez kell, hogy ne legyen üres tér a szavakat megelőzően
  %indentnext=no,% ez sem képes megakadályozni, a sortörést, ha a programkódban ki van hagyva egy sor
  %aligntitle=float,
  %command=\MyChapter,
  textdistance=0pt,%
  %commandbefore={\resetcounter[ParagraphNumber]\null},%
  %textcommand=\Szotextcommand,%
  %commandafter={\enspace\bullet\enspace},%
  %commandafter={\hskip 0.5em plus 0.25em minus 0.125em\relax\bullet\hskip 0.25em plus 0.25em minus 0.125em\relax},
  %commandafter={\hskip 0.5em plus 0.25em minus 0.125em\relax\bullet\kern 0.25em\relax},
  %commandafter={\kern 0.5em\bullet\kern 0.5em\relax},
  %alternative=text,%
  insidesection={\resetcounter[SzoBekSzam]},%
  style=\bf,%
  %textdistance=-\currentspaceskip%
]

%\defineenumeration
%  [Szo]
%  [
%    text=,%
%    number=no,%
%    title=yes,%
%    titlestyle=bf,%
%    titledistance=0pt,%
%    titleleft=,%
%    titleright={\enspace\bullet},%
%    distance=0.5em,%
%    alternative=serried%
%  ]

%\defineenumeration[axiomaenum]
%                  [
%                    before=,
%                    after=,
%                    text=T,%Space after Theorem is delibrate
%                    number=yes,%
%                    title=yes,%
%                    titledistance=0pt,%
%                    titleleft=,%
%                    titleright= axióma,%
%                    alternative=inmargin,%
%                    distance=1em,%
%                    titlestyle=\sc,%
%                    style=\em,% or italic
%                    width=fit,%
%                    headstyle=\tfx,
%                 ]
%
%\define\startaxioma
%  {\dosingleargument\dostartaxioma}
%
%\def\dostartaxioma[#1]%
%    {
%    %\ininner{\tfx\theaxioma}
%    \startaxiomaenum[#1]%
%     \crlf
%     }
%
%\define\stopaxioma
%    {\stopaxiomaenum
%    }




%\unexpanded\def\startBek{\incrementcounter[SzoBekSzam]%
%    \ifnum\rawcountervalue[SzoBekSzam]>1
%    \ininner{\tfx\rawcountervalue[SzoBekSzam]}%
%    \ignorespaces
%    \fi}

\def\BekRefNev{\currentstructuretitle.\rawcountervalue[SzoBekSzam]}
\def\BekRefKulcs#1{Bek:\currentstructuretitle:#1}
%def\BekRefNevesitettKulcs{Bek:\currentstructuretitle:\getvariable{SzoBekSzam:ref}{ref}}
\define\startBek{\dosingleempty\doStartBek}%
\def\doStartBek[#1]{\relax%
    %\doifelse{\rawcountervalue[SzoBekSzam]}{0}{\relax}{\relax}% az elsőnek meg kellene szüntetnie az új bekezdést
    \incrementcounter[SzoBekSzam]%
    %\setevariables[SzoBekSzam:ref][ref=xxx]
    \doifsomething{#1}{\setevariables[SzoBekSzam:ref][ref=#1]}%
    \setevariables[SzoBekSzam:ref][ref={\rawcountervalue[SzoBekSzam]}]%
    %\setvariables[SzoBekSzam:ref][%
    %  %ref=test%
    %  %ref={\iffirstargument#1\else{\rawcountervalue[SzoBekSzam]}\fi}%
    %  %ref={\doifsomethingelse{#1}{#1}{\rawcountervalue[SzoBekSzam]}}
    %  ref={}
    %  %ref={\rawcountervalue[SzoBekSzam]}
    %]%
    \expanded{\textreference[\BekRefKulcs{\rawcountervalue[SzoBekSzam]}]{\BekRefNev}}%
    \doifsomething{#1}{\expanded{\textreference[\BekRefKulcs{#1}]{\BekRefNev}}}%
    %\ifnum\rawcountervalue[SzoBekSzam]>1%
    \ininner{\tfx%
      \rawcountervalue[SzoBekSzam]
      %\BekRefKulcs{\rawcountervalue[SzoBekSzam]}
    }%
    \ignorespaces%
    %\fi
    }

%\unexpanded\def\stopBek{\relax}
\unprotect
\define\stopBek{\ctxlua{userdata.hivatkozasgyujto_ki()}\relax}
\protect


%\unexpanded\def\startAxioma{\incrementcounter[SzoBekSzam]%
%    \ifnum\rawcountervalue[SzoBekSzam]>1
%    \ininner{\tfx\rawcountervalue[SzoBekSzam]}%
%    \ignorespaces
%    \fi}
%
%\unexpanded\def\stopAxioma{\relax}

%\def\AxiomaCommand#1#2{#1 #2}
\def\AxiomaNumberCommand#1{\ininner{\tfx A#1}}
\def\AxiomaNumCommand{\ininner{A\somenamedheadnumber{Axioma}{current}}}
\def\AxiomaTextCommand#1{#1 axióma}

\definehead[Axioma][subsection]
% https://tex.stackexchange.com/a/375930/50554
%\setuphead[Axioma][%S
%  before=\relax,%
%  after=\relax,%
%  inbetween=\relax,%
%%  command=\AxiomaCommand,
%  commandbefore=\relax,%
%  commandafter=\relax,%
%  commandinbetween=\relax,%
%  beforesecton=\relax,%
%  aftersection=\relax,%
%  number=yes,%
%  numbercommand=\AxiomaNumberCommand,%
%  textdistance=1em,%
%  textcommand=\AxiomaTextCommand,
%  %commandbefore={\resetcounter[ParagraphNumber]\null},%
%  %commandafter={\enspace\bullet\enspace},%
%  %commandafter={\hskip 0.5em plus 0.25em minus 0.125em\relax\bullet\hskip 0.25em plus 0.25em minus 0.125em\relax},
%  %commandafter={\hskip 0.5em plus 0.25em minus 0.125em\relax\bullet\kern 0.25em\relax},
%  alternative={text},%
%  %insidesection={\resetcounter[SzoBekSzam]},%
%  numberstyle=\tfx,%
%  hang=broad,%
%  strut=no,
%  textstyle=\sc,%
%  style=\em,%
%  %textdistance=-\currentspaceskip%
%]

\defineenumeration[Axioma]
                  [
                    before=,%\blank,
                    after=,%\blank,
                    inbetween=,%\blank,
                    text=,%Space after Theorem is delibrate
                    number=yes,%
                    numbercommand=\AxiomaNumberCommand,%
                    title=yes,%
                    titledistance=0pt,%
                    titleleft=,%
                    titleright= axióma: ,%
                    alternative=serried,%
                    distance=0.5em,%
                    titlestyle=\sc,%
                    style=\em,% or italic
                    width=fit,%
                    headstyle=,
                 ]


%\defineenumeration[B]
%[location=inner,
%headstyle=bold,     % Bold remark headers
%inbetween={\blank[nowhite]},   % No blank line between remark header and remark
%after=\blank]

%\definemargindata[lasd]
%    [
%      distance=\outermargindistance,%
%      location=outer,%
%      align={nothanging,nohz,hyphenated,morehyphenated},
%      stack=continue, % or yes
%      style=\tfx\em,
%      width=\outermarginwidth
%    ]
\setupmargindata[inouter][
  align={outer,nothanging,nohz,hyphenated,morehyphenated},
  stack=continue,
  style=\tfx]

%\def\lasd#1{\inouter{\em #1}}
\unprotect
\define[1]\lasd{\ctxlua{userdata.hivatkozasgyujto_ad(\!!bs#1\!!es)}}
\define[1]\bibref{\ctxlua{userdata.bibrefgyujto_ad(\!!bs#1\!!es)}}
\protect

\setupindenting[no]


\definetabulate[Bib][|lw(2em)|p|]

\define[2]\bib
{\startBib
\NC #1
\NC {\em #2}
\NC\NR
\stopBib
}

\define[3]\Bib
{\startBib
\NC #1
\NC {\em #2}
\hfill\break
%\emptylines[1]
#3
\NC\NR
\stopBib
}

% Így lehet nagy betűs szöveget írni az első bekezdés elé
%\placefigure
%[left,none,high,low]{}
%{\framed[height=1cm]{Isten}}
