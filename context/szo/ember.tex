\startcomponent ember
\product szavak
\startSzo[title=ember,reference=szo:ember]
\startBek
A továbbiakban és másutt is helyenként nagybetűvel fogom írni az embert:
Az Ember a tökéletes ember.
Ember pedig kétféleképpen létezhet.
Először is ha Isten emberi alakot öltve jelenik meg, mely esetben születésétől fogva Emberről beszélek. \lasd{Isten}
Nem vagyok biztos benne, hogy Isten tesz ilyet.
Ha viszont mégis -- és számos szentírás nevez meg isteni avatárokat --, akkor az Ő általa megtesetesített Ember az adott kor és hely bizonyos típusú emberének megmutatja az igazságot, a helyes életet és az utat. \lasd{Krisztus}
Másodszor pedig Emberré válhat az ember is, ha tökéletesre csiszolt lélekkel a halála pillanatában az Isten otthonának kapuja előtt áll. \lasd{\in[Bek:Isten:masodik]}
Ez a szent Ember kijárta az anyagi világ iskoláját, minden ellentétpárt semlegesítve úrrá lett azokon és ezáltal eljutott a forráshoz -- Istenhez. \lasd{szent}
Éppen ezért ez utóbbi Embert szentnek vagy beérkezettnek is nevezhetem.
Szentté csak az ember válhat.
Az emberi testet öltött Isten nem szent ugyanis, hanem megmarad Istennek. \lasd{\in[Bek:Isten:4]}
\stopBek

\startBek
Már itt megfogalmazódott tehát az emberi élet célja, ami pedig kizárólag a szentté válás lehet.
\stopBek

\startBek
Amikor rendezem az ember helyét a világban, én magam is a helyemre kerülök.
Hasonlóan Istenhez, emberi mivoltunkat is a hit által ragadhatjuk meg, higgyünk bár az emberben mint Isten földi másában, avagy a darwini majomemberben.
Én az emberről a következőket hiszem:
\startitemize[n]
\item Eredet axióma: \emph{Az ember Isten szeretett gyermeke.}
\item Karmikus axióma: \emph{Az ember a karma hatása alatt áll.}
\stopitemize
\stopBek

\startBek
Az eredet axiómában hangsúlyos az Isten és az ember közötti szülő-gyermek viszony:
Megjelenik a személyesség, a személyes Isten és az ember közötti bensőséges ősi kapcsolat.
Ha saját magunknak személyiséget tulajdonítunk, akkor az meg kell legyen Istenben is, azaz Istennek kell legyen személyes aspektusa.
Az ember számára a személyes Isten tűnik a legkönnyebben megragadhatónak.
Az ember és az Isten közötti kapcsolat pedig a feltétel nélküli szeretet nyelvén valósulhat meg, ahogy a mély emberi kapcsolatok alapját is az ugyanilyen szeretet képezheti.
Isten, mint szülő részéről ez megvan minden ember felé.
\stopBek





Isten, szerető szülőként a gyermeki létből ifjúvá érett és világot látni kívánó ember számára kinyitotta az otthon, a paradicsom kapuját és kiengedte őt az anyagi világba, ahol aztán az ember az ellentétpárok hatása alá kerül, aminek következtében szinte bizonyosan letér az arany útról és elkóborol valamerre.
Persze pontosan ez is volt a célja, hogy tapasztatlatot, érettséget, de talán elsősorban vele egyenrangú társat, társakat szerezzen.
Az gyermeki vagy ifjú emberi lélek nem Ember, ahogy nem bölcs a gyermek sem aki önfeledt életet él.
Isten méltó utóda csak az Ember lehet.
Az Emberré váláshoz viszont el kell merülni az anyagban, meg kell tapasztalni az ellentétpárok világát.
Az Ember uralkodik önmagán és uralkodik az ellentétpárokon is.
Ez viszont tanult képessége amit csak az anyagi világban sajátíthat el.
Az ifjú emberi lélek ezzel a bölcsességgel nem rendelkezhet.
Isten számára bizonyosan bájos ő is, ám ahogy a legtöbb szülő, Isten is el kellett engedje gyermekét, hogy reményei szerint Emberként térjen majd vissza az Ő még nagyobb örömére.
A paradicsom embere tehát nem lehetett szent, amit a bűnbeesés is bizonyít.

Nem magára hagyatva kerül kiküldetésre az ember.
Isten az ember szívében székel:
Ez Isten éltető tulajdonságából is következik.
A szívben lakó Isten az ember döntései során érezteti számára, hogy mi a helyes választás.
Isten csendesen, minden tolakodás nélkül segít, ezért nagyon könnyű elengedni a jelentkező finom érzést és másként dönteni.
A szívünkben segítő Istent másként lelkiismeretnek is hívjuk.
A segítségen kívül azonban Isten figyelemmel is kísér minket, követ bennünket egész életünk során.
Figyelmét semmi sem kerülheti el, még legtitkosabb gondolataink sem.
Így lesz igazságos bíránkká halálunkat követően.

De nem csak Isten a társa az embernek:
Megjelent számára az ellenkező nem.
A két nem természete számos vonatkozásban ellentétes.
Az ellenkező nemhez való vonzalom igen erős, és segít abban, hogy kiegyensúlyozzunk bizonyos -- szintén az ellentétpárok körébe tartozó -- vonásokat azáltal, hogy megpróbáljuk megérteni a hozzánk közel álló másik nemhez tartozó lelket.
Így lehetnek a házastársak természetes segítői egymásnak az őshaza felé vezető úton, ott ugyanis a szerelem és a szeretet ereje is segíti őket. \lasd{szerelem} \lasd{szeretet}






közeledhessünk a ellentétes


Bármelyik nemhez is tartozzon az egyed, ez ellentétes nem

A végtelen jót és a végtelen rosszat is magába foglaló anyagi mezőn szerencsére nem könnyű az arany úttól túl messzire kallódni, ugyanis ebben megakadályoz minket a halál. \lasd{halál}
Így válik kvázi végessé a végtelen, amit amúgy fel nem foghatunk.
A halál visszaránt minket Istenhez, bárhol is járjunk éppen.
Ez Isten hatás axiómájából következik. \lasd{Isten}
A halál tehát egy újabb isteni útravaló.
Az egység felől nézve aztán lelkünk szembesülhet mindazzal, ami még nem forrott egységbe őbenne, és aminek megtapasztalása még hátra van.
A következő emberi útjára ennek ismeretében indul, ismét szabad kezet kapva. \lasd{lélekvándorlás}

A karma hatása alá tartozó Ember, emberi megnyilvánulásai során ki kell egyensúlyozódjék minden vonatkozó ellentétpár szerint, hogy visszatérhessen az isteni őshazába.

\stopSzo
\stopcomponent
