\startcomponent bevezetes
\startCim[title=BEVEZETÉS]

\startBek[oka]
Ez az írás egy személyes gondolatszedet, kutatás.
Eredeti célom az volt, hogy megértsenek, legalább azok, akik szeretnek.
Mára ez már csak álom, de van helyette másom: elég a rendeződés maga, elég ha összeszedem magam.
Korábban mindent össze-vissza beszéltem mert fogalmaim szanaszét hevertek fejemben.
Most azon vagyok, hogy egyszer minden fogalmam rendben legyen.
\stopBek

\startBek[rimesseg]
A forma a megjegyezhetőséget szolgálja.
Amúgy távol áll tőlem a líra.
A szöveg mégis valamelyest rímes, bár szedése folyószöveges.
Erre tanított az iskola, a verstanulásra.
Giccs a rím, de megtanulni magtanulam.
Jó dolog az egyszeri rendezettség, de önmagában az nem elég: véglegesen kell átvegye a káosz helyét.
Szükséges a mélyen bevésődött kaotikus fogalmi rendszertelenség felülírása, önmagam tudatos átprogramozása.
Ehhez pedig ismételni kell az újat, mantrázni új önmagamat, aki reményeim szerint érett ember lesz egyszer, szemben a mai gyerekkel.
A rímest könnyebb tanulni, a rímest könnyebb bevésni, a rímest jobb mantrázni.
Nincs művészi indíték, vagy ha mégis van, hátul áll a sorban.
Éppen ezért nem bíbelődök sokat, csak írom a sorokat, hogy haladjak, mert sok az anyag, sok fogalmam van, túlságosan.
\stopBek

\startBek[irodalmihivatkozasok]
Jó ideje sok ősi és értékes könyvet olvasok, és ezek az írások sok fogalom fundamentumát szépen megfogalmazták.
Gyakran olyasmi csúcsosodott ki tisztán és könnyedén, amit magam is tudtam önmagam legmélyén, csak fedte valami odabent, valami, ami nem engedte, hogy a dolgokat eredeti torzítatlan formájukban lássam.
És hogy az új fogalmi rendszerem ne alap nélkül legyen, az egyes szakaszok alatt meghivatkozom ezen írások vonatkozó részét egy nyilat követően ($\tfx\rightarrow$).
Ennek alakja a következő: a könyv címe rövidítve (a teljes címet a Bibliográfiában keresse), majd a megfelelő részre utalás általában számozott formában.
Ha a műnek van hagyományos hivatkozási rendszere, akkor magam is azt követem.
Ellenkező esetben fejezetet és bekezdést vagy részt, fejezetet, és bekezdést jeleznek a számok.
Új bekezdésnek veszek minden párbeszédes idézetet, felsorolási tagot, versszakot.
Nem használok oldalszámot, mert sokfélék lehetnek a kiadások.
\stopBek

\startBek[kereszthivatkozasok]
Egy rendszerben azonban a fogalmak nem önmagukban és elárvultan állnak, hanem másokkal összekapcsoltan, szoros egységben, szépen.
Sőt, valójában maga a rendszer a fontos.
Minden fogalom annak parányi alárendelt része; nem sokat tesz hozzá az egészhez, ám nélküle az nem jöhetne létre.
Kereszthivatkozások tömegével jelzem a fogalmi kapcsolatokat az egyes szakaszok alatt.
Eléjük kiteszek két párhuzamos vonalat ($\parallel$).
{\em Isten.6} a hivatkozás formája, így utalok Istenről alkotott fogalmam hatodik szakaszára.
A gyors keresést szolgálják az élőfej és a számozott szakaszok.
A fogalmak betűrendben állnak, ám ha kapitálisak, mint e bevezetés is, akkor a főszöveg megfelelő járulékos részére utalnak.
\stopBek

\stopCim
\stopcomponent
