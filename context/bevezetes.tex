\startcomponent bevezetes
\product fogalmak
\startCim[title=Rekurzivitás,reference=szo:rekurzivitás]
\startBek[fentlent]
Mióta felismertem a világ végtelenül szövevényes, szuperfraktálszerű rekurzivitását, megértettem a smaragdtábla állítását, miszerint ami fent, az lent.
\bibref{Ts 2{.}}
\stopBek

\startBek[isten_altal]
Isten mindenható, önmaga része, innen a világmindenség végtelen összetettsége, mélysége.
\lasd{\in[Bek:Isten:rekurzivitasa]}
\stopBek

\startBek[galaxis_pelda]
Látom a tejutat a Dráva örvényében, abban atomokat, az atomok körül keringő elektronokat, és ismét a tejutat.
\stopBek

\startBek[boly_pelda]
Kellő magasságból Róma is hangyaboly.
És amennyi az egyedi hangyának, annyi szerepe van a város lakójának, és e lakó agyában valamely neuronnak.
\bibref{GEB X.}
\stopBek

\startBek[irisz_pelda]
Látom az ősrobbanást nejem szemében, a szétáradó fényt a sötétség ellenében, mely sötét anyag ott van középen, pupilla néven, és ő azt látja, ami fény áttör e sötétségen.
\stopBek

\startBek[emberbolygo_pelda]
Azt mondják testemet milliónyi mikroorganizmus lakja: testem tehát e lények bolygója.
Ha pedig lélek költözhetett e testbe, akkor a Földnek is kell legyen teste és lelke.
A Föld is érez, ahogy én is teszem, ha testileg vagy lelkileg érintkezem.
\stopBek

\startBek[betegseg]
Ha egy sejt önálló utat kezd járni, rákos daganat bontakozik ki.
Istennel szakítva így lett az ember a föld betegsége, mert ahhoz még nincs elég bölcsessége, hogy saját útájt járva a természet hasznára váljon, és ne pedig ártson.
Ezért beteg maga is, mert amit tesz a maga szintjén, azt kapja a maga szintjén.
\bibref{SaC 1.6.19.}
\stopBek

\startBek[vertikalis_hasonlosag]
Ami fent, az lent: tulajdonság, állapot, hatás vagy folyamat a különböző szinteken eltéréssel ugyan, de azonosan van jelen.
Innen a következtetés egyikből a másikra, erről szól az asztrológia, és ez alapján állítom, hogy ha nekem van organizmusom, akkor a Földnek is van, sőt Istennek is lehet; ahogy ha nekem van személyiségem, akkor Istennek is kell legyen.
\lasd{\in[Bek:Isten:fiu]}
\bibref{SaC 4.1--2.}\bibref{Ts 2{.}}
\stopBek


\stopCim
\stopcomponent
