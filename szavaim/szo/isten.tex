\startcomponent isten
\product szavak
\startSzo[title=Isten,reference=szo:Isten]
%\textreference[r:Isten]{Isten}
\startBek[elso]
Istenhez fűződő viszonyunk csakis hit kérdése lehet.
Mégis, ha e hit néhány axióma elfogadására korlátozódik, akkor megóvhatom magamat attól, hogy belebonyolódjak a különféle tanok követésébe, melyek Isten és az ember közé dogmákat állítanak.
Melyek tehát a saját Isten-axiómáim?
\stopBek

\startAxioma[title=Mindenhatósági]
Isten mindenható.
\stopAxioma

\startAxioma[title=Hatás]
Isten hatása a teremtés, az éltetés és a pusztítás.
\stopAxioma

\startAxioma[title=Éltetési]
Az éltetésben Isten a közeg és a szikra.
\stopAxioma
%\startitemize[n]
%\item Mindenhatósági axióma: \emph{Isten mindenható.}
%\item Hatás axióma: \emph{Isten hatása a teremtés, az éltetés és a pusztítás.}
%\item Éltetési axióma: \emph{Az éltetésben Isten a közeg és a szikra.}
%\stopitemize

\startBek[masodik]
Mindenhatóság alatt tényleg \emph{mindent} értek.
A mindenható minden szférában -- lelki, szellemi, anyagi és minden szinteken -- mindenre hat, kivétel nélkül. \lasd{lélek} \lasd{szellem} \lasd{anyag}
A mindenhatósági axiómában az egyes számban mondott nagybetűs Isten az egységet és a személyes aspektust képviseli, a mindenhatóság pedig a végtelenséget és a személytelenséget.
A mindenhatóság pedig azért végtelenség, mert különben vagy nem lenne fogalmunk a végtelenre, vagy a mindenre hatás arra nem vonatkozna.
Isten tehát egyszerre egy és végtelen, egyszerre személyes és személytelen.
Ha Isten mindenre hat, akkor hat Istenre is: Isten tehát végtelenül rekurzív (a továbbiakban: mély). \lasd{rekurzív}
\stopBek

\startBek
Istenből ered minden, Isten mindennek az éltetője, és Isten a nyelője is mindennek.
A éltetési axióma alapján minden Istenben van és Isten is mindenben benne van.
Az anyagi világ is Isten emanációja, a teremtmények pedig Isten által élnek.
Minden, amire van fogalmunk, Isten legitim teremtménye, része, és áldozata, ugyanakkor magába is foglalja Istent.
Ebből következőleg minden rendelkezik a végtelen isteni mélységgel, így bizonyítást nyer Hermész Triszmegisztosz híres, ami fent úgy lent állítása. \lasd{smaragdtábla}
\stopBek

\startBek
Isten része az istentelenség is: Isten képes úgy kivonulni, hogy közben mégis ottmarad.
Isten úgy pusztít, hogy megtart, hiszen Istenből Istenbe tér minden, mi Isten által pusztul el.
\stopBek

\startBek
Ha felismertük a jó és a rossz és ezekkel együtt a semleges fogalmát, akkor Isten egyszerre mindhárom.
Isten tehát pozitív és negatív és semleges is egyszerre. \lasd{Sátán}
Sőt, a jó és a rossz végtelen szélességben és mélységben van meg Istenben.
A semlegességnek nincs dimenziója.
A mindenhatósági axióma szerint tehát a semlegesség Isten egységében, míg a jó és a rossz Isten végtelenségében van.
Hasonlóan, az éltetés az isteni egységben, míg a teremtés és a pusztítás az isteni végtelenségben vannak.
Ugyanígy kell lennie minden ellentétpárra és az azok közti határra is: ez utóbbi az isteni egységben van, a kettősség pedig már végtelenség.
Az egység mozdulatlan: nincs benne mozgástér.
Ez az eredő, a forrás, az egy Isten otthona.
Odakint a végtelen szabadság viszontagságos mezeje. \lasd{anyag}
E mező közepén arany út vezet az Isten otthonába -- haza. \lasd{őshaza}
\stopBek
\stopSzo
\stopcomponent
