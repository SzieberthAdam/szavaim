\startcomponent isten
\product szavak
\startSzo[title=Isten,reference=szo:Isten]
\startBek[orok_es_mindenhato]
Hiszem, és ebből indulok ki: {\em Isten örök és mindenható.} \lasd{hit}
Mindig volt és mindig lesz.
Mindenre hat Isten; amire ne hatna, olyan nincsen.
\stopBek

\startBek[szeret_es_eltet]
Hogyan hat Isten?
Ezt pedig így hiszem: {\em szeret és éltet.}
Ha Isten örök akkor az éltetés is örök.
Teremtés és pusztítás nem más, mint változás.
Születés és halál szintén csak változás.
Minden élt már korábban és élni fog aztán is.
Él a kő is: más volt hajdan és más lesz majdan.
Isten az eredő, Isten a nyelő, ki- és belélegző.
Így lüktet minden, így ver a szívem.
Nincs is más semmi, csakis az Isten.
\stopBek

\startBek[szamossaga]
Milyen az Isten?
Erre szó nincsen.
Így írja minden könyv és így mondja minden pap.
Értelemmel közelíteni, van annak értelme?
Isten egyes, kettős, hármas, vagy akárhányas?
Aki ilyet gondol, Istent nem látja mindenhol.
\stopBek

\startBek[rekurzivitasa]
Isten része minden megnyilvánulás és megnemnyilvánulás.
A mindenható önmagára is hat, végtelen mélységek vannak.
Végtelen szférák, végtelen síkok, mindenütt isteni mintázatok.
Minden Istenben van és mindenben Isten van.
Minden egészben van rész és minden részben ott az egész.
Örök a szeretet, de örök a változás: önmagára hatva még Isten is más.
\stopBek

\startBek[mintazat]
Lehet jobb nagyítóm, lehet jobb távcsövem, mégis az azonos mintákat figyelem.
Minden nagyon hasonló, de végtelen a változat.
Lehetetlen ezek számba vétele véges idő alatt.
Mindben ott a közös, az Ős, az Ís, az Isten.
Ha ezt beláttam, másra nincs szükségem.
\stopBek

\startBek[bent_es_kint]
Isten bennem és köröttem; önmagamban elérhetem.
Most a Földön vagyok, ember vagyok.
Bennem a szívem, bennem a lelkem, ezeket kell magamban kövessem.
Kint Isten nagyságát mutatja a természet végtelen csodája.
Most a Földön vagyok, ember vagyok.
Úgy kell élnem életemet, hogy betartom az elfeledett természettörvényeket.
\stopBek

\startBek[atya]
Elmerülni a békés rendbe, sodródni benne, és itt-ott, mint jámbor gyermek, tenni valamit Atyám örömére.
Ki éltetsz engem örökkön örökké, áldjalak Atyám most már mindörökké!
Gyermeked vagyok: te adod mindenhez a végső magot.
Minden a gyermeked, de ezt felismerni csak kevesek képesek.
Boldog lehetek, mert az éltető erő szerető Atyámból jő.
Boldog lehetek, mert az örök szeretet éltető Atyámból ered.
Boldog lehetek, hogy a felismeréshez -- mint ember -- kellő értelemmel rendelkezek.
\stopBek

\startBek[fiu]
Most a Földön vagyok, ember vagyok, küldetésben vagyok.
Ha én magam személy vagyok és Istenről tudok, Isten is személy.
A végtelen formát meg nem ragadhatom, Atyámat viszont könnyen imádhatom.
Fia vagyok viszont, nem a szolgája, mint derék fiú váljak Atyám hasznos szolgálatára.
Az Atya biztosította a helyet, hogy a teremtésbe belevihessem személyiségemet.
De mit adjak ha folyton elveszek, ha óvatos léptem nyoma is teremtmények sírja?
Így hogyan ékesítsem a természtet?
\stopBek

\startBek[viszonya]
Atyám teremt és Atyám pusztít.
Akkor vet és akkor arat, amikor az a legalkalmasabb.
Ő minden boldogság eredője, minden öröm osztója, minden csapás mérője, minden kór okozója, sorsunk szabója.
A sors kijár, a cserebogár marad sárga cserebogár.
Hiába keseregni, hiába örvendezni, mégis inkább örvendezzen mindenki.
Így csodás a mindig változó megnyilvánulás.
\stopBek

\startBek[pusztitas_okkal]
Lépjek és pusztítsak ha lépnem kell, de ok nélkül soha, mert olyat csak az ostoba tesz, és bűnéről nem tudván vállára újabb bűnt vesz.
A tudatlanság nem mentség, inkább vétek, a tudáshoz ugyanis nem az ész vezet, hanem a lélek.
Szeretet, tisztesség, jámborság és béke az eszességgel nem függenek össze.
Szeretet, tisztesség, jámborság és béke a tudó ember négy fő ismérve.
\stopBek

\startBek[oles_beavatkozas]
Mondom, szánt szándékkal nem szabad ölni, emberként ölés nélkül is meg kell tudnom élni.
A természet Isten által csodás, kerüljek hát minden szükségtelen beavatkozást!
\stopBek

\startBek[egyszeru_elet]
A paradicsomban mást nem kellett, csak gyümölcsért nyúlni.
Most mások pusztítása miatt muszáj magot is vetni.
Gyümölcsöt enni, kis hajlékban élni, alomszékre ülni, patakba mártózni, hamuval mosni, Istent imádni és minden mást szeretni, ebben asszonyt és gyermeket kitűntetni, a teremtésbe Atyám örömére finoman bekapcsolódni.
Kell egyebet tenni?
Ha igen, azt jól meg kell fontolni.
Bár így tudnék élni!
Bár így tudnánk élni!
Bár így tudna mindenki élni!
\stopBek

\startBek[rossz]
Isten része az istentelenség is: Isten kivonulhat miközben ottmarad mégis.
Mert Istenben van minden rossz és gonoszság ami valaha kiaknázatlanul állt.
Rendesen nem volt rossz, mert nem volt ki feltárja, és akkor sem lesz, ha mind aki bolygatja, e dolgát abbahagyja.
\stopBek

\startBek[valasztas]
Tudva indulhatok az isteni fénybe, vagy törhetek tudatlanul a végtelen sötétségbe.
\stopBek

\startBek[fenybe]
Anyagom súlya lefelé húz ha értelmem fel nem emel.
Így küzd bennem anyag és értelem, és kezdetben sajnos az anyag van fölényben.
Testem a gépem, önműködő gépem, miért irányít engem?
Nem vagyok még készen.
Minden anyagi kihágás után újabb megerősítés kell, hogy nem úgy, hanem ésszel.
Százszor, ezerszer szenved vereséget az értelem, hogy aztán végül felülkerekedhessen -- végérvényesen.
\stopBek

\startBek[sotetsegbe]
Most pedig épphogy emelkedvén, vagy lassabban ereszkedvén látom, hogy sokan mily nagy buzgósággal törnek lefelé a mélybe, és viszik a maguk lángját elfojtani a sötétségbe.
Versengve keresik az újabb fertelmeket magukra véve a későbbi gyötrelmeket.
Hova süllyedtem magam is, ijesztő belegondolni is!
\stopBek

\startBek[sorscsapasok]
Piszkos bűneim után kesereghetek a sors valamely csapásán?
Ha az élet gondtalan, bűnhődni kell majd egy másikban.
Korábbi életeim mérlegét nem ismerem, kell e feltételezett bűnök bocsánatát kérnem?
Nem hiszem.
Legjobb vezeklés a csendes beletörődés.
Azért viszont tehetek, hogy mostani bűneimért későbbi életekre ne rójak terheket.
Ahogy Atyámhoz közeledek, tiszta fénye leégeti rólam az elkövetett bűnöket.
\stopBek

\startBek[lefele_ne_tovabb]
Tovább viszont semmiképpen sem süllyedhetek, mert valahol mélyen örök kárhozat volna osztályrészem.
De jó, hogy nemrég -- talán ez volt az utolsó esélyem -- felnéztem!
\stopBek

\startBek[himnusz]
Szóljon a himnusz, az ima, a mantra: mind az Atyát magasztalja!
Magasztalás közben nem süllyedünk aljasságba.
Emelkedj értelem, kerüld a sötétet!
Imádd ember az Urat, és megleled fényed.
\stopBek

\startBek[sors]
Minden hatás Istené hát.
Isten szabja az emberi sorsot.
Azt kérded, hol a szabadság?
Mondom, ahol kiszabatott.
\stopBek


\stopSzo
\stopcomponent
