\startcomponent isten
\product szavak
\startSzo[title=Rekurzivitás,reference=szo:rekurzivitás]
\startBek[fentlent]
Mióta felismertem a világ végtelenül szövevényes, szuperfraktálszerű rekurzivitását, megértettem a smaragdtábla állítását, miszerint ami fent, az lent.
\bibref{Ts 2{.}}
\stopBek

\startBek[isten_altal]
Isten mindenható, önmaga része, innen a világmindenség végtelen összetettsége, mélysége.
\lasd{\in[Bek:Isten:rekurzivitasa]}
\stopBek

\startBek[galaxis példa]
Látom a tejutat a Dráva örvényében, abban atomokat, az atomok körül keringő elektronokat, és ismét a tejutat.
\stopBek

\startBek[bóly példa]
Kellő magasságból Róma is hangyaboly.
És amennyi az egyedi hangyának, annyi szerepe van a város lakójának, és e lakó agyában valamely neuronnak.
\bibref{GEB X.}
\stopBek

\startBek[írisz példa]
Látom az ősrobbanást nejem szemében, a szétáradó fényt a sötétség ellenében, mely sötét anyag ott van középen, pupilla néven, és ő azt látja, ami fény áttör e sötétségen.
\stopBek

\startBek[emberbolygó példa]
Azt mondják testemet milliónyi mikroorganizmus lakja: testem tehát e lények bolygója.
Ha pedig szellem költözhetett e testbe, ne volna akkor a Földnek szelleme?
\stopBek

\startBek[gaia példa]
Látom a segget, ha egy vulkánba nézek.
Látom az ejtett sebeket, a külszíni fejtésket.
A kútfúrók szúnyogok, a gazdák ekcémát okozó gombák vagy baktériumok.
\stopBek

\startBek[betegség]
Ha egy sejt önálló utat kezd járni, rákos daganat bontakozik ki.
Istennel szakítva így lett az ember a föld betegsége, mert ahhoz még nincs elég bölcsessége, hogy saját útjával ne ártson, de használjon.
\bibref{SaC 1.6.19.}
\stopBek

\startBek[vertikális hasonlóság]
Ezért beteg maga is, mert amit tesz a maga szintjén, azt kapja a maga szintjén.
Ami fent, az lent, és ez igaz nem csak az állapotokra, de a folyamatokra is. % ezt írom; javíani
Ha lecsapsz egy legyet, ne lepődj meg, ha téged is leütnek!
\stopBek


\stopSzo
\stopcomponent
