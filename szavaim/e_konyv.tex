\startenvironment e_konyv

% Betűkészlet
\setupbodyfont[libertinus,9pt]

% Magyar nyelvű.
\mainlanguage[hu]
% Bekezdések (Gyurgyák, 78--79. o.).
\setupalign[
  justified,
  nothanging,
  nohz,
  hyphenated,
  morehyphenated,
  tolerant]
\setupbodyfontenvironment[default][em=italic]

%\setuphyphenation[method=traditional]
\setuphyphenation[method=default]

\setupinmargin[style=\em]

\setuplayout[%
  grid=yes,% Soregyen (Gyurgyák 319. o.).
  height=\dimexpr27cc+\headerheight+\headerdistance+\footerheight+\footerdistance\relax,%
  location=middle,%
  width=18cc,%
  backspace=\dimexpr0.5\dimexpr\paperwidth-\makeupwidth\relax\relax,%
  topspace=\dimexpr\dimexpr\paperheight-\makeupheight\relax/3\relax,%
  leftmargindistance=0.5cc,%
  rightmargindistance=0.5cc,%
  leftmargin=1.5cc,%
  rightmargin=1.5cc
]

% Soregyennél a nagy ékezetes soroknál üres sorok keletkeznek,
% ha az alábbi nincs beállítva.
% (https://tex.stackexchange.com/questions/343299/when-i-activate-grid-layout-in-context-empty-lines-inexplicably-appear)
\setupinterlinespace[height=0.75,depth=0.25]

% Lábjegyzetek
\setupnotation[footnote][
  alternative={standard},
  indenting=yes% Tömbös szedés. (Gyurgyák 124--125. o.)
]
\setupnote[footnote][
  margindistance=2em,
  rule=off,% Lábjegyzet sor kihagyásával, léni nélkül.
          % (Gyurgyák 123. o.)
  % Szélessége a szedéstükörrel megegyező legyen:
  width=\ifdim\hsize<\makeupwidth\hsize\else\makeupwidth\fi
]

% Oldalszámozás
\setuppagenumbering[%
  location={footer},% Tükör alá. (Gyurgyák 203. o.)
  %location={footer, right},
  %alternative=doublesided,
]

%\setupparagraphnumbering[%
%  state=start,%
%  style=\em,%
%  distance=0pt
%]

% Definiálok egy lapméretet a körülvágott könyvnek
% (Gyurgyák, 401. o).
\definepapersize[%
  VB6% vágott B5
  ][%
  height=163mm,%
  width=119mm%
]
% Beállítom a körülvégott méretet lapméretnek.
\setuppapersize[VB6]

\stopenvironment
