\startcomponent hit
\product szavak
\startSzo[title=hit,reference=szo:hit]
\startBek[hit_def]
A hit az elfogadott magyarázat.
Tág értelmezésben e magyarázat a a hívő világképére és a világban betöltött szerepére vonatkozik.
\stopBek

\startBek[hit_kontra_tudas]
A hit nemtudás, hiszen amit másoktól veszek át, azt csak hihetem, de nem tudhatom. %\irod{sorsmintesely}
A tudás megszerzése hosszú és fáradalmas, ezzel szemben a hit gyors és kényelmes. %\lasd{tudás}
A hit ugyanakkor a tudás előfeltétele, a lehetségesnek tartás, ami ebben a minőségében a már elsajátított tudásra is támaszkodik.
\stopBek

\startBek[hitbol_a_tudasba_mas]
A hitből a tudásba vezető út számos buktatót rejt.
Gyakran hamar kiderül, hogy a végső tudás egészen más lesz, mint amire először számított az ember.
Ez a felismerés megfutamíthatja, és hitében is megingathatja.
Ha az illető kitart a hite mellett, akkor a tudatlanság mellett tart ki.
\stopBek

\startBek[hitbol_a_tudasba_ellentetes]
A kutató a legritkább esetben nem képes találni a hitével ellentétben álló és kellően megalapozott elméletet vagy elméleteket. %\lasd{kutatás}
Ez megingathatja hitében.
\stopBek

\startBek[hitbol_a_tudasba_kozos]
Az is lehet viszont, hogy ezek az elméletek közös alapokon nyugszanak, mely esetben a hit redukciója és erősödése következhet be, annak ellenére, hogy a számosság nem az igazság mércéje.
\stopBek

\startBek[hitbol_a_tudasba_hitismeret_altal]
Ráadásul a másoktól átvett kapcsolódó gondolatok, lévén, hogy szintén hitek, csak a hitismeretet mélyíthetik, tudáshoz viszont nem vezetnek. %\lasd{hitismeret}
A mélyebb hitismeret is a tudás felé tett lépés, ám mégis egy folyamatosan tornyosuló akadálya a tudás megszerzésének azáltal, hogy a tudás színében látszik tündökölni.
\stopBek

\startBek[hitbol_a_tudasba_igazabol]
Tudáshoz csak az ősrendhez és a személyes Istenhez való közeledés által nyert tisztánlátás juttat. \lasd{hazaindult}\lasd{hazatért}
\stopBek

\stopSzo
\stopcomponent
