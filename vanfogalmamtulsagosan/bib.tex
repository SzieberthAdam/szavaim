\startcomponent bibliografia
\product szavak

\setuptabulate[
    split=yes,
    frame=off,
    after={\relax},
    before={\relax},
  ]

\startsubject[title=Bibliográfia]

% A listát megelőző szöveget ide kell írni

\blank

\bib
{1Móz}
{Mózes első könyve / Teremtés könyve}

\Bib
{Bg}
{Bhagavad-gíta / A magasztos szózata}
{
A védikus irodalom alapműve, a hindu vallás legszentebb könyve.
Legkésőbb a Kr{.} e{.} 3{.} században keletkezett.
Benne az legfőbb isteni személyiség, Krisna a világnak, az ember sorsának és szerepének, valamint az üdvösség jóga általi elérésének átfogó tanítását adja.
}

\Bib
{CH}
{Corpus Hermeticum}
{
A hermetikus tanok központi műve, Hermész Triszmegisztosznak tulajdonított tanítások.
Felfehetőleg a Kr{.} u{.} 1--3{.} századból származó filozófiai szövegek gyüjteménye, melyek görög nyelven maradtak fent, és a korai keresztény egyházra is komoly hatást gyakoroltak.
A szövegek eredetiségét a Nag Hammadiban talált lelet igazolta.
}

\Bib
{GEB}
{{\em Douglas R{.} Hofstadter:} Gödel, Escher, Bach}
{
A kognitív tudomány egyik fontos írása, melyben az intelligens megismerés és annak határai, az egymázba ágyazódott sémák ereje nehéz, ám mégis szórakoztató formában kerül bemutatásra.
}

\bib
{Luk}
{Lukács evangéliuma}

\Bib
{SaC}
{{\em Thorwald Dethlefsen:} Schicksal als Chance / A sors mint esély}
{
A sors helyénvalógágát és az ezotéria lényegét bemutató könyv, mely rámutat a világban uralkodó vertikális mintázatok, ősprincípiumok létére, és arra, hogy ezek miként határozzák meg a sorsot, és hogyan tud az ember harmóniába kerülni a mindenséggel.
}

\Bib
{Ts}
{Tabula smaragdina / Smaragdtábla}
{
A Kr{.} e{.} 6{.} századra tehető szövege hellenisztikus forrásból maradt ránk.
Az eredeti táblát a legenda szerint Hermész Triszmegisztosz egyiptomi sírjában, múmiája mellett találták.
}

\stopsubject
\stopcomponent
